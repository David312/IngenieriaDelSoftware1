%%% Local Variables:
%%% mode: latex
%%% TeX-master: "IS1apuntes"
%%% End:
\section{Arquitectura Software}
\label{sec:arquitectura:arquitectura}


\begin{description}
\item[Componente] Bloque del sistema. Parte que combinas con la arquitectura.
\item[Servicio] Funcionalidad que los componentes proporcionan a los actores.
\end{description}

Al dividir un sistema en componente, hay que definir los servicios que
proporciona cada componente.

\subsection{Estilos}
\label{sec:estilos}

El estilo es la forma general de un sistema, semejante a lo que serían
los patrones de diseño (\ref{sec:arquitectura:diseñoestructura}). Al
definir un estilo, se deben especificar los elementos como los bloques
básicos de contrucción, las conexiones entre los bloques y las reglas
que espeficican cómo se combinan los servicios.

\begin{figure}[h]
  \centering
  \begin{tabular}{l | l}
    \textbf{Técnica}&\textbf{Patrón}\\\hline
    Abstracción&Niveles \\
    Encapsulación & Expedidor-receptor \\
    Ocultación de información&Reflexión, Composite \\
    Modularización&Niveles, Pipes \& Filters, Composite\\
    Acoplamiento y
    cohesión&Publicador-Suscriptor,Cliente-Despachador-Servidor\\
    Separación de intereses&Modelo-Vista-Controlador
  \end{tabular}
  \caption{Patrones que ayuan a aplicar técnicas}
  \label{fig:patronesestilo}
\end{figure}

\subsection{Índice de un documento de arquitectura}
\label{sec:indicearquitectura}
\begin{itemize}[noitemsep]
\item Objetivos
\item Requerimientos (\emph{funcionales, no funcionales})
  (\ref{sec:arquitectura:requerimientos})
\item Decisiones y justificación
\item Modelo conceptual (\ref{sec:documentacion})
  \begin{itemize}[noitemsep]
  \item Modelo de componentes lógicos
  \item Modelo de procesos
  \item Modelo físico
  \item Modelo de despliegue
  \end{itemize}
\item Despliegue de la arquitectura
\end{itemize}

Otra información relevante del documento de arquitectura es presentar
distintos diagramas:
\begin{itemize}[noitemsep]
\item Diagrama de Clases (\emph{Lógica}).
\item Diagrama de Paquetes (\emph{Desarrollo}).
\item Diagrama de Interacción (\emph{Procesos}).
\item Diagrama de Despliegue (\emph{Física}).
\end{itemize}

\subsubsection{Pasos en la identificación de un problema}
Metas del proceso\textrightarrow Recogida de
información\textrightarrow Conceptos de la Arquitectura\textrightarrow
Cliente de la solución\textrightarrow Definición del problema

\subsection{Patrones}
\label{sec:patrones}

\begin{enumerate}[noitemsep]
\item Especificar el problema.
  \begin{itemize}[noitemsep]
  \item Dividir el problema.
  \item Encontrar el contexto.
  \item Considerar pros/cons.
  \item Acceder al catálogo de patrones.
  \end{itemize}
\item Seleccionar la categoría de los patrones (\emph{arquitectónicos
  o de diseño}).
\item Categoría del problema.
\item Comparar descripciones del problema.
\item Comparar beneficios y compromiso.
\item Elegir la mejor variante.
\end{enumerate}

Entre los ejemplos de patrones están: \emph{N-Niveles, Filtros y
  Tuberías, Pizarra, Modelo-Vista-Controlador}.