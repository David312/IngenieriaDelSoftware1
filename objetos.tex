% \setlength{\parindent}{4em}
% \setlength{\parskip}{1em}
%%% Local Variables:
%%% mode: latex
%%% TeX-master: "IS1apuntes"
%%% End:

\section{Orientación a objetos}
\label{sec:org3620eb4}
\subsection{Enfoque estructurado}
\label{sec:orgc590b61}
Se denomina enfoque estructurado a la forma de pensar el software en
términos de funciones de transformación de datos (se disocia entre
funciones y datos, y las tareas se interpretan como una transformación
de los últimos).

\subsubsection{Ejemplo: Pintar un círculo}
\label{sec:orge1ddd3a}
El enfoque estructurado resuelve el problema de pintar un círculo de
la siguiente forma:
\begin{itemize}
\item Usa una definición de círculo que esté acorde con los recursos
  de software (en este caso la expresión algebraica).

\begin{equation}
     R^{2} \le (x - x_{0})^{2} + (y - y_{0})^{2}
\end{equation}

donde el radio R y las coordenadas del centro son las constantes que
especifican un círculo concreto.
\item Disocia la definición de círculo en dos partes y las
  reinterpreta:
\begin{itemize}
\item Considera que R y el centro son datos para pintar el círculo y
  añade el color.
\item Convierte la expresión declarativa en una función operativa que
  transforma el conjunto de datos precedentes en (x, y, color) de
  todos los píxeles para pintar el círculo en la pantalla.
\end{itemize}
\item Como resultado final se obtiene un sistema capaz de pintar un
  círculo en términos de un proceso de transformación de datos.
\end{itemize}

El sistema software se expresa como una función F(x) que transforma el
conjunto de datos (R, x\(_{\text{0}}\), y\(_{\text{0}}\)) en otro
conjunto de datos, en este caso de píxeles.

(insertar figura 1.2)

A este tipo de esquema se le denomina \emph{diagrama de flujo de
  datos}. \textbf{El diagrama de flujo de datos es un esquema
  asíncrono} (no expresa secuencias); las flechas sólo indican los
flujos de datos, no el orden de ejecución.


El principal problema del enfoque estructurado es latente en el
momento en el que queremos añadir más elementos e interactuar con
ellos, por ejemplo, pintar varios círculos y actuar sobre los mismos
de forma selectiva, digamos borrar el segundo que se pintó.  Podríamos
hacer un bucle para crear n círculos, pero si queremos guardarlos
tendríamos que añadir tantas variables como círculos, con el objetivo
de retener cada conjunto de constantes. Este sistema es una
duplicación del sistema para solo un caso.


La disgregación de los conceptos en datos y funciones tiene sus pros y
sus contras, por ejemplo, este enfoque permite trabajar directamente
con la idea de base de datos o archivo, lo cual puede ser
beneficioso. Sin embargo, esta disociación implica \textbf{disminuir
  nuestro nivel de abstracción}.
\subsection{Enfoque orientado a objetos}
\label{sec:orgab1dc13}
El enfoque orientado a objetos es la forma particular de pensar el
software en términos de elementos que colaboran entre sí para realizar
tareas.  Este enfoque nos da un nivel de abstracción superior al
estructurado, asociando cada elemento del problema a un elemento
software.  Cada elemento software tiene las propiedades íntegras de
cada elemento del discurso (lo que \emph{hace a una cosa ser una
  cosa}).
\subsubsection{Ejemplo: Pintar un círculo}
\label{sec:orge1a8962}
En el ejemplo anterior, definimos un objeto \texttt{círculo} que
cumple las propiedades de un círculo según la definición que hemos
adaptado para nuestro sistema (en este caso, el objeto contiene un
centro y un radio) y tiene los mecanismos para pintarse y crearse como
elemento.

El enfoque de objetos piensa:
\begin{itemize}
\item En variables software capaces de recordar las constantes de un
  círculo, capaces de pintar un círculo y capaces de crearse a sí
  mismas como variables.
\item En el sistema software en términos de la interacción de estas variables, dadas sus respectivas capacidades para ejecutar operaciones, es decir, cómo
relacionar todas las variables para conseguir que se realice la tarea de pintar círculos.
\end{itemize}

(introducir figura 1.8)

Este esquema muestra el sistema software, donde se aprecian las
relaciones entre las variables software que ejecutan la tarea de
pintar un círculo.  Como vemos, el sistema software, aun aplicando el
mismo algoritmo, tiene una organización diferente, y por tanto sus
propiedades también varían.
\subsubsection{Objetos}
\label{sec:org2b984f4}
Esto que hemos ido llamando \emph{variables software} se conocen en
este enfoque como \textbf{objetos}, y amplían la idea de la variable
software tratada en el enfoque estructurado, ya que tienen capacidad
de expresar cualquier cosa, incluso operaciones.  Otra definición
complementaria de objeto es la siguiente: \emph{Un objeto es un
  elemento software cualitativamente distinto capaz de expresar un
  concepto más amplio, más ambiguo: cosa}

Los objetos interactúan entre ellos mediante \textbf{mensajes},
solicitudes a objetos para que ejecuten operaciones.

